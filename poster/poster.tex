%%%%%%%%%%%%%%%%%%%%%%%%%%%%%%%%%%%%%%%%%
% Jacobs Landscape Poster
% LaTeX Template
% Version 1.1 (14/06/14)
%
% Created by:
% Computational Physics and Biophysics Group, Jacobs University
% https://teamwork.jacobs-university.de:8443/confluence/display/CoPandBiG/LaTeX+Poster
% 
% Further modified by:
% Nathaniel Johnston (nathaniel@njohnston.ca)
%
% This template has been downloaded from:
% http://www.LaTeXTemplates.com
%
% License:
% CC BY-NC-SA 3.0 (http://creativecommons.org/licenses/by-nc-sa/3.0/)
%
%%%%%%%%%%%%%%%%%%%%%%%%%%%%%%%%%%%%%%%%%

%----------------------------------------------------------------------------------------
%	PACKAGES AND OTHER DOCUMENT CONFIGURATIONS
%----------------------------------------------------------------------------------------

\documentclass[final]{beamer}

\usepackage[scale=1.0]{beamerposter} % Use the beamerposter package for laying out the poster
\usepackage[acronym,toc]{glossaries}
\input{../acros}
\usetheme{confposter} % Use the confposter theme supplied with this template

\setbeamercolor{block title}{fg=dblue!80,bg=white} % Colors of the block titles
\setbeamercolor{block body}{fg=black,bg=white} % Colors of the body of blocks
\setbeamercolor{block alerted title}{fg=white,bg=dblue!70} % Colors of the highlighted block titles
\setbeamercolor{block alerted body}{fg=black,bg=dblue!10} % Colors of the body of highlighted blocks
% Many more colors are available for use in beamerthemeconfposter.sty

%-----------------------------------------------------------
% Define the column widths and overall poster size
% To set effective sepwid, onecolwid and twocolwid values, first choose how many columns you want and how much separation you want between columns
% In this template, the separation width chosen is 0.024 of the paper width and a 4-column layout
% onecolwid should therefore be (1-(# of columns+1)*sepwid)/# of columns e.g. (1-(4+1)*0.024)/4 = 0.22
% onecolwid should therefore be (1-(# of columns+1)*sepwid)/# of columns e.g. 
% (1-(3+1)*0.025)/3 = 0.3
% Set twocolwid to be (2*onecolwid)+sepwid = 0.464
% Set threecolwid to be (3*onecolwid)+2*sepwid = 0.708

\newlength{\sepwid}
\newlength{\onecolwid}
\newlength{\twocolwid}
\newlength{\threecolwid}
\setlength{\paperwidth}{36in} % A0 width: 46.8in
\setlength{\paperheight}{48in} % A0 height: 33.1in
\setlength{\textwidth}{34in} % A0 width: 46.8in
\setlength{\textheight}{46in} % A0 height: 33.1in
\setlength{\sepwid}{0.025\paperwidth} % Separation width (white space) between columns
\setlength{\onecolwid}{0.3\paperwidth} % Width of one column
\setlength{\twocolwid}{0.625\paperwidth} % Width of two columns
\setlength{\threecolwid}{0.95\paperwidth} % Width of three columns
\setlength{\topmargin}{-0.5in} % Reduce the top margin size
%-----------------------------------------------------------

\usepackage{graphicx}  % Required for including images
\newcommand{\Cyclus}{\textsc{Cyclus}\xspace}%
\usepackage{tabularx}
\newcolumntype{b}{X}
\newcolumntype{s}{>{\hsize=.5\hsize}X}
\newcolumntype{m}{>{\hsize=.75\hsize}X}
\newcolumntype{z}{>{\hsize=.65\hsize}X}

\usepackage{booktabs} % Top and bottom rules for tables
\usepackage{xspace}
\usepackage{amsmath}
\usepackage{exscale}
\usepackage[labelformat=simple]{caption}


\setbeamertemplate{bibliography item}[text]




%----------------------------------------------------------------------------------------
%	TITLE SECTION 
%----------------------------------------------------------------------------------------

\title{%
  \texorpdfstring{%
    \makebox[\linewidth]{%
      \makebox[0pt][l]{%
        \raisebox{\dimexpr-\height+\baselineskip}[0pt][0pt]
          {\includegraphics[height=2.5\baselineskip]{UIUC_Logo}}% Left logo
      }\hfill
      \makebox[0pt]{Comparison Between Continuous and Batchwise}%
      \hfill\makebox[0pt][r]{%
        \raisebox{\dimexpr-\height+\baselineskip}[0pt][0pt]
          {\includegraphics[height=3.3\baselineskip]{arfc_atom}}% Right logo
      }%
    }%
  }
  % To make a title with two lines, remove the "%" from the following two lines 
  {Comparison Between Continuous and Batchwise}
  {Online Reprocessing in Serpent2}
  {\vspace{1cm}}
  } % Poster title

\author{\textbf{Luke Seifert}, Madicken Munk}
\institute{University of Illinios at Urbana-Champaign, Department of Nuclear, Plasma, and Radiological Engineering, Urbana, IL 61801}
%----------------------------------------------------------------------------------------

\begin{document}

\addtobeamertemplate{block end}{}{\vspace*{2ex}} % White space under blocks
\addtobeamertemplate{block alerted end}{}{\vspace*{2ex}} % White space under highlighted (alert) blocks

\setlength{\belowcaptionskip}{2ex} % White space under figures
\setlength\belowdisplayshortskip{2ex} % White space under equations

\begin{frame}[t] % The whole poster is enclosed in one beamer frame

\begin{columns}[t,totalwidth=\threecolwid] % The whole poster consists of three major columns, the second of which is split into two columns twice - the [t] option aligns each column's content to the top

\begin{column}{0.5\sepwid}\end{column} % Empty spacer column

\begin{column}{\onecolwid} % The first column

%----------------------------------------------------------------------------------------
%	INTRODUCTION
%----------------------------------------------------------------------------------------

\begin{block}{Introduction}

\textbf{Molten Salt Reactor Online Reprocessing}

\begin{itemize}
	\item Depletion of Molten Salt Reactors requires accounting for reprocessing
	\item Batchwise modeling of Molten Salt Reactors is common \cite{rykhlevskii_modeling_2019, betzler_molten_2017}
	\item Continuous modeling offers unique advantages over batchwise modeling
\end{itemize}

\vspace{0.7em}
\textbf{Comparison of Methods}

\begin{itemize}
	\item An identical toy model is implemented for both methods
	\item Continuous model uses varying number of steps
	\item Multiple approaches are implemented for the continuous model
	%\item Potential weaknesses of continuous model are investigated
\end{itemize}


\begin{figure}
	\label{fig:toy_geom}
	\includegraphics[width=0.6\linewidth]{images/CR0_geom1.png}
	\caption{Geometry of toy model used in serpent2 for continuous and batchwise reprocessing models.}
\end{figure}



\begin{table}[H]
\renewcommand{\arraystretch}{1.25}
\caption{Approach Acronyms and Descriptions}
\label{tab:res_key}
\begin{center}
\begin{tabular}{ | c | c | }
 \hline
        Approach & Description\\
 \hline
 \hline
 	SP pre & SaltProc steady batch pre-depletion\\
	SP post & SaltProc steady batch post-depletion\\
	CR & Cycle Rate continuous approach\\
	SPCR & SaltProc Cycle Rate continuous approach\\
	CTD & Cycle Time Decay continuous approach\\
	CTRL & Control method (no reprocessing feeds or removal)\\
 \hline
\end{tabular}
\end{center}
\end{table}

\end{block}

%----------------------------------------------------------------------------------------
%	OBJECTIVES
%----------------------------------------------------------------------------------------

%This section creates an orange border around a white box
\setbeamercolor{block alerted title}{fg=black,bg=norange} % Change the alert block title colors
\setbeamercolor{block alerted body}{fg=black,bg=white} % Change the alert block body colors
\begin{alertblock}{Objectives}
\begin{itemize}
        \item Capture the precise differences in continuous and batchwise models
	\item Determine effective depletion step sizes for continuous reprocessing
	\item Investigate validity of using average feed rates during depletion
	%\item Investigate potential weaknesses of continuous reprocessing
\end{itemize}

\end{alertblock}



\setbeamercolor{block alerted title}{fg=white,bg=dblue} % Change the alert block title colors
\setbeamercolor{block alerted body}{fg=black,bg=white} % Change the alert block body colors

\begin{alertblock}{Contact Information}
	\setbeamercolor{block title}{fg=norange,bg=white} % Change the block title color
	\begin{itemize}
		\item Email: \href{mailto:seifert5@illinois.edu}{seifert5@illinois.edu}
		\item Phone: \href{tel:18652790603}{+1 865 279 0603}
	\end{itemize}
% These specific elements are optional, but you should have some method for people to contact you.	
	
\end{alertblock}





%----------------------------------------------------------------------------------------
%	Programs
%----------------------------------------------------------------------------------------

%\begin{block}{Program 1}
%Description
%
%\begin{figure}
%	\label{fig:figure_label_ex1}
%	\includegraphics[width=0.9\linewidth]{graphic_name1}
%	\caption{Caption1 Reference \cite{call_tag_article}}
%\end{figure}
%
%\end{block}
%
%\begin{block}{Program 2}
%
%Description
% 
%\begin{itemize}
%	\item{Item 1}
%	\item{Item 2}
%	\item{Item 3}
%\end{itemize}
%
%\end{block}
%
%----------------------------------------------------------------------------------------

\end{column} % End of the first column

\begin{column}{\sepwid}\end{column} % Empty spacer column


%----------------------------------------------------------------------------------------

\begin{column}{\onecolwid} % The second column
%----------------------------------------------------------------------------------------
%	MODELS
%----------------------------------------------------------------------------------------

\begin{block}{Reprocessing Models}
\vspace{0.7em}
\textbf{Batchwise Reprocessing}

\begin{itemize}
        \item Iteratively perform depletion with external adjustments
	\item \textbf{\emph{Steady Batch}} uses small removals each depletion step.

\begin{align}
	X = \frac{1}{T_{cyc}}
\end{align}
	\begin{itemize}
		\item $T_{cyc}$ is the cycle time
		\item $X$ is the fractional removal rate
	\end{itemize}

	%\item Small removals each depletion step is Steady Batch
	\item \textbf{\emph{Bulk Batch}} uses full removal after a set number of depletion steps.
	%\item Full removal after set number of steps is Bulk Batch
	\item SaltProc is used to run batchwise reprocessing for Serpent2
	\item The current version of SaltProc uses the Steady Batch method
\end{itemize}


\begin{table}[H]
\renewcommand{\arraystretch}{1.25}
\caption{Batchwise Reprocessing Methods}
\label{tab:batch_methods}
\begin{center}
\begin{tabular}{ | c | c | c | c | }
 \hline
	Approach & Cycle Time & X [s$^{-1}$] & Step Removal\\
 \hline
 \hline
	Bulk Batch [3d] & 20 s & - & 1\\
	Bulk Batch [3d] & 30 d & - & \, 0$^{*}$ \\
	Steady Batch [3d] & 20 s & 3.86E-6 & 1\\
	Steady Batch [3d] & 3 d & 3.86E-6 & 1\\
	Steady Batch [3d] & 30 d & 3.86E-7 & 0.1\\
 \hline
\end{tabular}
\end{center}
\end{table}
	\begin{center}
\footnotesize{$^{*}$ Bulk removal occurs after 30 days, so the step fractional removal becomes 1 at 30 day step.}\\
	\end{center}

\vspace{0.7em}
\textbf{Continuous Reprocessing}

\begin{itemize}
	\item Adds "decay-like" term to Bateman equation, less iterative \cite{aufiero_extended_2013}

%\begin{align}
%\frac{dN_j}{dt}_{base} = \sum_{i \neq j}  \left [ \left( \gamma_{i \rightarrow j} \sigma_{f, i} \Phi + \lambda _{i \rightarrow j} + \sigma_{i \rightarrow j} \Phi \right) N_i \right ] - \left ( \lambda_j + \sigma_j \Phi \right ) N_j
%\end{align}

\begin{equation} \hfill
\begin{split}
%\frac{dN}{dt} = \sum_j \lambda_j N_j + \gamma \Sigma_f \phi + \Sigma_k \phi - \lambda N - \Sigma \phi - C
\frac{dN_j}{dt}_{base} = \sum_{i \neq j}  & \left [ \left( \gamma_{i \rightarrow j} \sigma_{f, i} \Phi + \lambda _{i \rightarrow j} + \sigma_{i \rightarrow j} \Phi \right) N_i \right ]\\
 & - \left ( \lambda_j + \sigma_j \Phi \right ) N_j
\end{split}
\hfill\label{eq:Bateman_default} \end{equation}




\begin{align}	
\frac{dN_j}{dt}_{net} = \frac{dN_j}{dt}_{base} -  \lambda_{r, j} N_j + \sum_{mat} \lambda _{r, i \rightarrow j} N_i
\end{align}


The symbols given in the equations are defined as follows:
\begin{itemize}
\item $N_j$ is the atomic density of isotope j.
\item $\gamma_{i \rightarrow j}$ is the fractional fission product yield of $j$ in the fission of isotope $i$.
\item $\sigma_{f, i}$ is the microscopic fission cross section of isotope $i$.
\item $\Phi$ is the spectrum-averaged scalar flux in the fuel region.
\item $\lambda _{i \rightarrow j}$ is the decay constant of decay $i \rightarrow j$.
\item $\sigma_{i \rightarrow j}$ is the microscopic transmution cross section of reaction $i \rightarrow j$.
\item $N_i$ is the atomic density of isotope $i$.
\item $\lambda_j$ is the decay constant of isotope $j$.
\item $\lambda_{r, j}$ is the reprocessing constant for removal of isotope $j$.
\item $\sigma_j$ is the microscopic total transmutation cross section of isotope $j$.
\item $\lambda _{r, i \rightarrow j}$ is the reprocessing constant for feed of material $i \rightarrow j$.
\end{itemize}


	\item \textbf{\emph{Cycle Time Decay}} (CTD) model treats reprocessing as decay


\begin{align}
	\tau_{1/2} = \frac{1}{2} T_{cyc}
\end{align}

\begin{align}
	\lambda_r = \frac{ln(2)}{\tau_{1/2}}
\end{align}

	\item \textbf{\emph{Cycle Rate}} (CR) treats as linear fractional removal, same as Steady Batch

\begin{align}
	\lambda_r = ln \left( \frac{1}{1-X} \right)
\end{align}


	\item \textbf{\emph{SaltProc Cycle Rate}} (SPCR) mimics batchwise reprocessing with continuous model
\end{itemize}


\begin{table}[H]
\renewcommand{\arraystretch}{1.25}
\caption{Continuous Reprocessing Methods}
\label{tab:cont_methods}
\begin{center}
\begin{tabular}{ | c | c | c | c | c | }
 \hline
	Approach & Cycle Time & $\tau_{1/2}$ & X [s$^{-1}$] & $\lambda_{r}$ [s$^{-1}$]\\
 \hline
 \hline
 Cycle Time Decay & 20 s & 10 s & - & 6.93E-2\\
 Cycle Time Decay & 3 d & 1.5 d & - & 5.35E-6\\
 Cycle Rate & 20 s & - & 0.05 & 5.13E-2\\
 Cycle Rate & 3 d & - & 3.86E-6 & 3.86E-6\\
 SaltProc Cycle Rate & 20 s & - & 3.86E-6 & 3.86E-6\\
 SaltProc Cycle Rate & 3 d & - & 3.86E-6 & 3.86E-6\\
 SaltProc Cycle Rate & 30 d & - & 3.86E-7 & 3.86E-7\\
 
 \hline
\end{tabular}
\end{center}
\end{table}



%\vspace{0.7em}
%\textbf{Model Overview}
%
%\begin{table}[H]
%\renewcommand{\arraystretch}{1.25}
%\caption{Different Reprocessing Approaches}
%\label{tab:keff_vals}
%\begin{center}
%\begin{tabular}{ | c | c | c | c | c | }
% \hline
%	Approach & Cycle Time & $\tau_{1/2}$ & X [s$^{-1}$] & $\lambda_{r}$ [s$^{-1}$]\\
% \hline
% \hline
% Bulk Batch [3d] & 20 s & - & - & -\\
% Bulk Batch [3d] & 30 d & - & - & -\\
% Steady Batch [3d] & 20 s & - & 3.86E-6 & -\\
% Steady Batch [3d] & 3 d & - & 3.86E-6 & -\\
% Steady Batch [3d] & 30 d & - & 3.86E-7 & -\\
% \hline
% Cycle Time Decay & 20 s & 10 s & - & 6.93E-2\\
% Cycle Time Decay & 3 d & 1.5 d & - & 5.35E-6\\
% Cycle Rate & 20 s & - & 0.05 & 5.13E-2\\
% Cycle Rate & 3 d & - & 3.86E-6 & 3.86E-6\\
% SaltProc Cycle Rate & 20 s & - & 3.86E-6 & 3.86E-6\\
% SaltProc Cycle Rate & 3 d & - & 3.86E-6 & 3.86E-6\\
% SaltProc Cycle Rate & 30 d & - & 3.86E-7 & 3.86E-7\\
% 
% \hline
%\end{tabular}
%\end{center}
%\end{table}


\end{block}

%----------------------------------------------------------------------------------------

\end{column} % End of column 2

\begin{column}{\sepwid}\end{column} % Empty spacer column

\begin{column}{\onecolwid} % The third column

\begin{block}{Results}
%\textbf{Mass Balancing}
%
%\begin{figure}
%	\label{fig:mass_bal}
%	\includegraphics[width=0.9\linewidth]{images/simple_mass_bal.png}
%	\caption{Continuous reprocessing mass balance for single fissile feed and two isotope system with $\tau_{1/2}$ of 10 seconds.}
%\end{figure}
%
%\vspace{0.7em}
%\textbf{Comparison Results}

%\begin{figure}
%	\label{fig:keff_30d_batch}
%	\includegraphics[width=0.45\linewidth]{images/cumulative_keff_batch.png}
%	\includegraphics[width=0.45\linewidth]{images/cumulative_keff_coont.png}
%	\caption{Continuous and batch models $k_{eff}$ over time when using the matching depletion steps and feed rates and when continuous uses a single step and average feed rates.}
%\end{figure}

%\begin{itemize}
%	\item \textbf{\emph{SP pre}} is the SaltProc steady batch results pre-depletion (post-reprocessing)
%	\item \textbf{\emph{SP post}} is the SaltProc steady batch results post-depletion (pre-reprocessing)
%	\item \textbf{\emph{CR}} is the Cycle Rate continuous approach
%	\item \textbf{\emph{SPCR}} is the SaltProc Cycle Rate continuous approach
%	\item \textbf{\emph{CTD}} is the Cycle Time Decay continuous approach
%	\item \textbf{\emph{CTRL}} is a control approach with no reprocessing feeds or removal
%\end{itemize}



\textbf{Multiple Steps}
\begin{figure}
	\label{fig:keff_30d_batch}
	\includegraphics[width=0.9\linewidth]{images/cumulative_keff_batch.png}
	\caption{Continuous and batch models $k_{eff}$ over time when using the matching depletion steps and feed rates.}
\end{figure}

\textbf{Single Step}
\begin{figure}
	\label{fig:keff_30d_batch}
	\includegraphics[width=0.9\linewidth]{images/cumulative_keff_coont.png}
	\caption{Continuous and batch models $k_{eff}$ over time when continuous uses a single step and average feed rates.}
\end{figure}

\begin{table}[H]
\renewcommand{\arraystretch}{1.25}
\caption{$k_{eff}$ at 30 Days for 3 and 30 Day Steps}
\label{tab:keff_vals}
%\begin{center}
\begin{tabular}{ | c | c | c | c | }
 \hline
 Approach & 3d Step $k_{eff}$ & 30d Step $k_{eff}$ & Diff [pcm]\\
 \hline
 \hline
 CR & 0.622815 & 0.622043 & 77\\
 SPCR & 0.612871 & 0.611481 & 140\\
 CTD & 0.62241 & 0.623246 & 84\\
 CTRL & 0.594924 & 0.595784 & 86\\

 \hline
\end{tabular}
%\end{center}
\end{table}



\end{block}



\setbeamercolor{block alerted title}{fg=black,bg=norange} % Change the alert block title colors
\setbeamercolor{block alerted body}{fg=black,bg=white} % Change the alert block body colors
\begin{alertblock}{Future Work }
\begin{itemize}
		\item Mass balancing of continuous reprocessing for full reactor
		\item Comparison of models for full reactor
		\item Depletion step size development over reactor lifetime
\end{itemize}

\end{alertblock}

%% This section creates an box with an orange border and a white background
%\setbeamercolor{block alerted title}{fg=black,bg=norange} % Change the alert block title colors
%\setbeamercolor{block alerted body}{fg=black,bg=white} % Change the alert block body colors
%\begin{alertblock}{Future Work }
%\begin{itemize}
%		\item Future work
%\end{itemize}
%
%\end{alertblock}


%----------------------------------------------------------------------------------------
%	ACKNOWLEDGEMENTS
%----------------------------------------------------------------------------------------

\setbeamercolor{block title}{fg=norange,bg=white} % Change the block title color

\begin{block}{Acknowledgements}

	This material is based upon work supported under an Integrated University Program Graduate Fellowship. The authors are grateful for this generous support.

The authors thank Kathryn Huff for her contribution and support of this work in its early stages.
	
\end{block}

%----------------------------------------------------------------------------------------
%	CONTACT INFORMATION
%----------------------------------------------------------------------------------------


\begin{block}{References}

	{\footnotesize\bibliographystyle{abbrv} 
	\bibliography{poster}}
\end{block}


%----------------------------------------------------------------------------------------



\end{column} % End of the third column

\end{columns} % End of all the columns in the poster

\end{frame} % End of the enclosing frame

\end{document}
\begin{column}{\sepwid}\end{column} % Empty spacer column
